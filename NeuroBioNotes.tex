\documentclass[12pt, a4paper]{article}

\usepackage{listings}
\usepackage{color}
\usepackage{enumitem}
\usepackage{amsthm}
\usepackage{amssymb}
\usepackage{listings}
\usepackage{setspace}
\usepackage{graphicx}
\graphicspath{ {./figures/} }


\title{Introduction to Neurobiology}
\author{William Darko}
\date{Summer 2021}

\pagenumbering{arabic}

\begin{document}

\maketitle
\newpage

\tableofcontents

\newpage

\section{About this course}
\paragraph*{}
Objective of the course is to learn how the nervous system produces behaviour,
how we use our brains in our day to day lives, and how neuroscience can help explain
the problems afflicting people today. There'll be focus on functional human neuroanatomy,
and neuronal communication, to help understand how we perceive the world, do body movements,
and interact with others.

\newpage

\section{Resources}

\begin{itemize}
    \item \textbf{Coursera: Understanding the Brain: The Neurobiology of Everyday Life}
    taught by professor of Neurobiology Peggy Mason, at the University of Chicago (https://www.coursera.org/learn/neurobiology)
\end{itemize}

\newpage

\section{Introduction}
\begin{itemize}
    \item \textbf{The Diving Bell and the Butterfly}: 
    Jean-Dominique Bauby, locked-in syndorme.
\end{itemize}

\section{The Nervous System}
\subsection{The Four Functions}
The locked-in syndrome tells of the four basic functions of the brain/central nervous system.
\begin{enumerate}
    \item \textbf{Voluntary Movement:} Every thing we do that is driven by the brain,
    both deliberate actions, such jumping, speaking, raising your hand, etc, and not so deliberate
    actions like wincing in reaction to stepping on a lego piece.
    \item \textbf{Perception:} Perception is distinct from sensation; its what we conciously
    appreciate about sensation. Its what we're capable of being aware of such as vision, hearing,
    smell, taste, balance, position, lung pressure, etc.
    \item \textbf{Homeostasis:} Used to keep body within its physiological limits.
    For example, making sure the body has enough oxygen, the right blood pressure, right body temperature.
    Also, homeostasis accounts for life cycle events like a mother giving birth, and the conditions 
    needed for the child to be healthy. Altogether, a process of maintaining healthy internal conditions.
    \item \textbf{Abstract functions:} Higher functions of the central nervous system like
    thinking, language, motivation, feeling emotion, etc. Also, plays a huge role in how we interact with other humans.
\end{enumerate}

\subsection{Central Anatomy}
Mapping of the four functions to regions of the brain.
\begin{enumerate}
    \item Motor neurons which exist in the brain stem, or the spinal cord are responsible
    for \textbf{voluntary movement}. There are less than 100,000 motor neurons, out of about 200 billion
    in the entire nervous system. Motor neurons in the brain system are responsible of movement of the mouth, face,
    hence speech, facial expressions, swallowing, etc.
    Motor neurons in the spinal cord are responsible for bodily movements like movement of the arms,
    legs, etc.
    \item \textbf{Perception} is entirely in the \textbf{Forebrain}; more specifically,  it depends entirely on the \textbf{Cerebral Cortex}.
    Perception is one of the higher brain functions; if the information carried by neurons does not 
    make it to to the Cerebral Cortex, then there's no perception; there's no concious appreciation/awareness
    of sensation.
    \item \textbf{Homeostasis} depends on the \textbf{Forebrain, brain stem, and spinal cord}.
    The forbrain's contribution to homeostasis is hormonal. The brain stem has a varied contribution; its responsible for
    the automatic changes we're not able to control, the autonomic (involuntary) functions of our nervous system. The spinal cord's
    contribution is similar to that of the brain stem.
    
    The brain stem, and spinal cord serve as pathways for information, both incoming, and outgoing. 

    \item \textbf{Abstract functions} are entirely in the forebrain, and function independent of the brain stem, and spinal cord.
    The forebrain is the "seat of conciousness"; all perception, and abstract conginitve functions like memory, depend on the forebrain;
    more specifically,the cerebral cortex.
\end{enumerate}

{
    \centering
    \includegraphics[width=8cm]{BrainImg_FromSide_SidesAnnotated.png}
    \includegraphics[width=8cm]{BrainImg_CutFromSide_MidHindBrainStem.png}

}
\newpage

\section{Neurons; the ``stars" of the nervous system"}

\subsection{Parts of the Neuron}
\paragraph*{}
There are four parts to neurons.
\begin{enumerate}
    \item Cell body, also known as the \textbf{Soma}. Place that keeps the cell going,
    makes all the materials needed for the entire neuron
    \item \textbf{Dendrites}; they branch out of the cell body, creating a \textbf{tree like structure
    called the dendritic arbour/tree}. They're responsible for gathering information for the neuron. Information 
    goes into the dendrites. Dendrites may be perceived as the ears of the neuron.
    \item Infomation processed locally from dendrites, sent out through one \textbf{axon} which is more gloablly distributed
    compared to the dendrites. Axon can travel a metre; and ultimately carry information to a \textbf{synaptic terminal}.
    \item \textbf{Synaptic terminals} are the point of information transfer between cells; infromation is carried to the terminal
    via axons. There's a small space between the synaptic terminal, and the receiving cell/dendrite; that space is where the
    event of information transfer occurs, which we know as a \textbf{synapse}.
\end{enumerate}

\subsection{Neuronal Uniqueness}
\paragraph*{}
A wide variety of ways the anatomy of neurons can make them different from each other.
Neurons also differ in the sense of what the neurons are connected to, what neurons are talking to; the inputs
and outputs. In addition to the \textbf{anatomy}, other differences include:
\begin{itemize}
    \item \textbf{Excitability}: how talkative is the neuron. How much work is needed
    to get neuron to fire action potentials. How likely or unlikely is it to fire action potentials.
    \item \textbf{Neurotransmitters}: What chemical/substance does the neuron use to communicate.
    For instance, some neurons use serotonin. How does the neuron ``speak''. Difference in communication speed.
    Affirmative vs negative.
\end{itemize}

\subsection{Glial cells}
\paragraph*{}
Neuron don't exist on their own, they requrie the support of Glial Cells. There's a one-to-one
mapping of neurons to glia. There's different types of Glial Cells:
\begin{enumerate}
    \item \textbf{Astrocytes}: behave as sanitation workers of the brain. They collect the refuse of neurons, such as excess ions, and 
    neurotransmitters. They also allow neurons to get to where they have to go during development. Synapses are envoloped in the processes of 
    Astrocytes, which helps with maintaining them. Comprice about 20 percent of glial cells
    \item \textbf{Oligodendrocytes (Central nervous sys.), and Schwann cells (Peripheral nervous sys.)}: Create
    myelin in their respective areas of the nervous system. Combined, these comprice about 75 percent of glial cells.
    \item \textbf{Microglia}: comprice about 5 percent of glia. Immune cells from the blood lineage that have invaded into the central nervous system,
    and are idle as long as the human body is health. However they react to areas of damage, and sometimes even contribute to the damage. Implicated in several diseases like
    chronic pain, and neurodegenerative disorders like alzheimer's.
\end{enumerate}

\subsection{Myelin}
\paragraph*{}
Meylin is a \textbf{fatty wrap that goes around some axons.} The difference between a myelinated axon, and an
unmeylinated axon/naked axon.
\begin{itemize}
    \item \textbf{Unmyelinated} axons can only transfer information at a slow rate; about
    \textbf{0.2 - 1.0 m/s}
    \item \textbf{Myelinated} axons have myelin applied and carry infromation orders of magnitude faster than naked axons.
    the speed at which these axons carry information increases to \textbf{2 - 120 m/s}. A speed inperceptible to us humans.
\end{itemize}
Myelinated axons are significant especially in cases where we need infromation fast such as infromation about our balance, which we need the fastest. Neurons
that support a posture against gravity, use myelinated axons.

The infomation that gets transferred through myelinated axons can be perceived as binary digits
\textbf{(011001110111...)}. Whats important about this sequence of binary digits is less about whether its a 1 0r 0, but 
more of the \textbf{pattern of 1's} that appear in the sequence. That is the neural code; the \textbf{1s represent
action potentials or spikes}. The \textbf{timing} of the spikes is what carries information.

The spikes that traverse through the myelinated axon travel so fast because they jump between the gaps of
the myelin wraps, thus not having to traverse through the wraps, and hence the entire physical
distance of the axon; effectively shortening the distance, and time required for the action potental to travel.
\\

{
    \centering
    \includegraphics[width=10cm]{myelinated_axon_whiteboard.png}

    \includegraphics[width=10cm]{action_potential_myelinated_axon_animated.png}
    
}

\paragraph*{}
However, consider if along the myelinated axon, there becomes less, and less myelin wraps.
Our initial action potential, if starting as a string of bits say \textbf{001100111010110111101}, may come out on the other end
due to the decrease in myelin wraps as a completely incoherent message inconsistent with the initial one, say: \textbf{000001001000100000100}.
This is caused by what are known as \textbf{demyelinating diseases} which degrade the information transfer.

\subsection{Demyelinating Diseases}
\paragraph*{}
Recall that Glial Cells that create Myelin are either in central nervous system, or Peripheral nervous system. Thus people
who get demyelinating diseases either get it in the CNS, or PNS, but not both.

So the problem is either in the interaction between the \textbf{Oligodendrocytes} and the axon,
in which case we get a \textbf{Central Demyelinating Disease}, which the most common by far is \textbf{Multiple Sclerosis}.

Or, there's a problem in the Schwann cells, and its interaction between with the axons. In which case,
we get something like \textbf{Charcot-Marie-Tooth} which are a diverse group of heriditary demyelinating neuropathies.

Neural code is disrupted any where there is demyleination. Demyleination would mostly affect
motor axons which travel information the fastest. Symptoms someone gets from Multiple Sclerosis would
depend on which axon group is affected.


\end{document}